% Based on Models 1 and 2 in "Searching" by Clif Kussmaul

Hi-Lo is a number guessing game with simple rules, played by school children.
\begin{enumerate}[nosep]
\item There are two players -- $A$ and $B$.
\item Player $A$ thinks of a number from 1 to 100.
\item Player $B$ guesses a number.
\item Player $A$ responds with \\ ``too high'', ``too low'', or ``you win''.
\item Players $B$ and $A$ continue to guess and \\ respond until $B$ wins (or gives up).
\end{enumerate}

\quest{10 min}

\Q How many different answers can player $A$ give? \ans{Three (too high, too low, you win).}

\Q When does the game end?

\begin{answer}[1em]
The game ends when $B$ guesses the correct number or gives up.
\end{answer}


\Q Play the game once to ensure that everyone understands the rules.

\Q Identify 4--5 different guessing strategies that Player $B$ could use.
Each strategy should describe a \textbf{different approach} to the game.
For example: \textit{Start at 1, and count up until the correct answer is found.}
In computer science, we call such strategies \textbf{algorithms}.
Try to have a mixture of simple and clever algorithms, including ones that young children could use.

\begin{enumerate}
\item \ans{Randomly guess numbers until you get the right one.}
\item \ans{Start at 100 and count down by 1 to the correct answer.}
\item \ans{Count up by 10 until too high, and then count down by 1.}
\item \ans{Count up by 25, then count down by 5, and finally count up by 1.}
\item \ans{Start at 50, if too high search 1--49, or if too low search 51-100, etc.}
\end{enumerate}


\Q Rank order the algorithms with regard to how \textbf{fast} they will find the right answer.
Write 1 for the fastest algorithm (fewest guesses) and 5 for the slowest one (most guesses).

\begin{answer}
In this particular example:

1: e ~ 2: d ~ 3: c ~ 4: b ~ 5: a
\end{answer}
